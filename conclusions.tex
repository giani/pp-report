Read-copy update (RCU) is a synchronization mechanism that gets heavily used in the linux kernel. It improves scalability by allowing readers to execute concurrently with writers. RCU ensures read coherence by maintaining multiple versions of data structures and ensuring that they are not freed until all pre-existing read-side critical sections complete. Using RCU for developing parallel system is complicated and requires lot of care. The user need to follow many rules or assumption and incorrect usage of rcu leads to programming bugs. These bugs are hard to debug. There are existing solution mention in section which deals with the incorrect usage of RCU but they are imprecise and cause many false positive.  In this project we are trying to analyse the incorrect usage of RCU violating three rules mentioned in section~\ref{sec:proposal}. Our system uses \emph{Watchpoint} mechanism implemented using Dynamic Binary Instrumentation (DBI) technique to track the references of RCU protected data. \emph{Watchpoint} mechanism provides us infrastructure to track the memory references and gives us callback on the access of RCU protected data which is used to verify the access policy. The algorithm we used to implement the policy is simple and uses thread and watchpoint \emph{Generation number} to check the policy.  Our system updates \emph{Generation number} when it encounters rcu primitive. Our system currently enforces simple policies and is limited to classical RCU only. 
