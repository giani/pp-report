Section~\ref{sec:back} mentions that usage of RCU requires a lot of care. 
There are many rules that must be followed while writing code
which uses RCU. Figures~\ref{fig:rcuderefbug} and~\ref{fig:rcuusebug} show
some examples.

\begin{figure}[h]
\centering
\begin{lstlisting}
f() {
	...
	rcu_read_lock();
	p = q;
	do_something(p);
	rcu_read_unlock();
	...
}
\end{lstlisting}
\caption{RCU bugs: Not using rcu\_derefence}\label{fig:rcuderefbug}
\end{figure}

\begin{figure}[h]
\centering
\begin{lstlisting}
f() {
	...
	rcu_read_lock();
	p = rcu_dereference(q);
	x = p;
	rcu_read_unlock();
	...
	rcu_read_lock();
	do_something(x);
	rcu_read_unlock();
	...
}
\end{lstlisting}
\caption{RCU bugs: Not using in the correct critical section}\label{fig:rcuusebug}
\end{figure}

In figure~\ref{fig:rcuderefbug} a reference to q takes place directly without the use
of \emph{rcu\_dereference}. Since \emph{rcu\_dereference} has not been used to
dereference q, the compiler barriers are not in place and therefore the compiler
is free to aggresively speculate pointer values. This may lead to an undesirable
result.

In figure~\ref{fig:rcuusebug}, while p has been obtained with the
use of \emph{rcu\_dereference}, it is used (in the form of x) after the critical section
has been announced. As the end of a read critical section has been announced,
RCU is now free to reclaim the reference that p may be pointing to. As a result
when p is dereference in the next critical section, that can be a pointer to
garbage which is very undesirable.  A similar bug would be if the reference
obtained was outside the critical section.

Both these examples be broadly class as RCU pointer leaks.

As mentioned in section~\ref{sec:back}, there are many rules that must be followed
while using RCU. For the purposes of this report, we shall name two rules.
\begin{itemize}
\item{\bf Rule 0}: Any access to RCU protected data must happen in an RCU read critical section.
\item{\bf Rule 1}: Any reference to RCU protected data in a read critical section must obtained with the use \emph{rcu\_dereference}
\item{\bf Rule 2}: Any reference obtained \emph{using rcu\_dereference} must be used within the critical section it was obtained in.
\end{itemize}

We shall call the bugs of the type represented in figure~\ref{fig:rcuderefbug} as \emph{Rule 1 violations}
and those in figure~\ref{fig:rcuusebug} as \emph{Rule 2 violations}.
This project aims to detect and report bugs of these two types.
Detecting these bugs is extremely challenging. Due to the nature of RCU implementation,
bugs are subtle and at times hard to reproduce. An example is that it is possible for the
references to be illegally obtained, but still be valid.

In order to detect \emph{Rule 1 violations}, we must first identify data
that has been protected by RCU. We must then be able to track every
dereference to it. We must also be in a position to instrument
every \emph{rcu\_dereference()} call.

For \emph{Rule 2 violations}, we must know when a dereference takes place.
At that point in time, we need to be able to check if the reference has
been obtained legally. If so, we must know if it has been obtained in
the current critical section.





%
%In order to detect these issues, we use \emph{generations}. A \emph{Generation}
%can be defined as an RCU read critical section. A generation has the following
%properties

%\begin{itemize}

%\item \emph{Generations} are per-thread
%\item A new \emph{generation} starts when \emph{rcu\_read\_lock} is called
%\item A \emph{generation} ends when \emph{rcu\_read\_unlock} is called

%\end{itemize}

%To detect these violations, we use \emph{granary} which is described in
%further detail in section~\ref{sec:impl}. Granary provides us the ability
%to add unlimited watchpoints.

%We first identify RCU protected data.
%This can be acheived in a number of ways. The easiest way is to intercept calls
%to \emph{rcu\_assign\_pointer}. A watchpoint is added at that point. The watchpoint
%is \emph{per-pointer}. The watchpoint is erased only when a reclaim takes place.

%Whenever an \emph{rcu\_dereference} takes place, a check is made to confirm that
%a \emph{generation} has started. If not, this detects that an \emph{rcu\_dereference}
%has taken place outside a critical section and is a \emph{rule 2 violation} If a
%generation has started, we then tag the associated RCU pointer with the current
%generation.

%When a dereference of RCU protected data takes place, the very first check is made
%to see if a \emph{generation} has started. The absence of a generation detects
%a \emph{rule 2 violation}.

%Next the RCU pointer must be checked to see if it was legally obtained. This is checked by checking the
%\emph{generation} information of the pointer. Since \emph{generation} information
%is setup only at the point when an \emph{rcu\_dereference} takes place, absence
%of this information indicates a \emph{rule 1 violations}.

%However the presence of the \emph{generation} information does not imply that the RCU
%pointer is legal. In order to confirm that, the \emph{generation} in the RCU pointer is
%compared with current thread's \emph{generation}. If the two generations are the same,
%then the RCU pointer is legal. Any other case denotes a \emph{rule 2 violation}
