As mentioned in Section~\ref{sec:back} usage of RCU requires a lot of care in
the use of RCU. There are many rules that must be followed while writing code
which uses RCU. Figures~\ref{fig:rcuderefbug} and~\ref{fig:rcuusebug} show
some examples.

\begin{figure}[h]
\centering
\begin{lstlisting}
f() {
	...
	rcu_read_lock();
	p = q;
	do_something(p);
	rcu_read_unlock();
	...
}
\end{lstlisting}
\caption{RCU bugs: Not using rcu\_derefence}\label{fig:rcuderefbug}
\end{figure}

\begin{figure}[h]
\centering
\begin{lstlisting}
f() {
	...
	rcu_read_lock();
	p = rcu_dereference(q);
	x = p;
	rcu_read_unlock();
	...
	rcu_read_lock();
	do_something(x);
	rcu_read_unlock();
	...
}
\end{lstlisting}
\caption{RCU bugs: Not using in the correct critical section}\label{fig:rcuusebug}
\end{figure}

In figure~\ref{fig:rcuderefbug} a reference to q takes place directly without the use
of \emph{rcu\_dereference}. In figure~\ref{fig:rcuusebug}, while p has been obtained with the
use of rcu\_dereference, it is used (in the form of x) after the critical section
has been announced. A similar bug would be if the reference obtained was outside the
critical section. These bugs can be broadly classed as RCU pointer leaks.

As mentioned in section~\ref{sec:back}, there are many rules that must be followed
while using RCU. For the purposes of this report, we shall name two rules.
\begin{itemize}
\item{\bf Rule 1}: Any reference to RCU protected must obtained with the use \emph{rcu\_dereference}
\item{\bf Rule 2}: Any reference obtained \emph{using rcu\_dereference} must be used with the critical section it was obtained in.
\end{itemize}

We shall call the bugs of the type represented in figure~\ref{fig:rcuderefbug} as \emph{Rule 1 violations}
and those in figure~\ref{fig:rcuusebug} as \emph{Rule 2 violations}.
