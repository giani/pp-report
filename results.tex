As part of evaluation, we used rcu torture test. It provides support for testing all rcu implementations and gets enabled by config option CONFIG\_RCU\_TORTURE\_TEST. It creates an rcutorture kernel module that can be loaded to run a torture test.  The test periodically outputs status messages via printk(), which can be examined via dmesg. The test is started when the module is loaded, and stops when the module is unloaded. We verified our system by running it in a Guest virtual machine Hosted by system equipped with an Intel(R) Core(TM) i7-860X 2.80 GHz CPUprocessor and natively natively on a 2.93GHz Intel Core ™ 2 Duo CPU. In the course of evaluation all system was running Linux 2.6.32 kernel.

\subsection{Hypothesis}
Our evaluation strategy aims to test three hypotheses:
\begin{itemize}
	\item  qwe
	\item  wer
\end{itemize}
